%%%%%%%%%%%%%%%%%%%%%%%%%%%%%%%%%%%%%%%%%
% Beamer Presentation
% LaTeX Template
% Version 1.0 (10/11/12)
%
% This template has been downloaded from:
% http://www.LaTeXTemplates.com
%
% License:
% CC BY-NC-SA 3.0 (http://creativecommons.org/licenses/by-nc-sa/3.0/)
%
%%%%%%%%%%%%%%%%%%%%%%%%%%%%%%%%%%%%%%%%%

%----------------------------------------------------------------------------------------
%	PACKAGES AND THEMES
%----------------------------------------------------------------------------------------

\documentclass{beamer}

\mode<presentation> {
	
	% The Beamer class comes with a number of default slide themes
	% which change the colors and layouts of slides. Below this is a list
	% of all the themes, uncomment each in turn to see what they look like.
	
	%\usetheme{default}
	%\usetheme{AnnArbor}
	%\usetheme{Antibes}
	%\usetheme{Bergen}
	%\usetheme{Berkeley}
	%\usetheme{Berlin}
	%\usetheme{Boadilla}
	%\usetheme{CambridgeUS}
	%\usetheme{Copenhagen}
	%\usetheme{Darmstadt}
	%\usetheme{Dresden}
	%\usetheme{Frankfurt}
	%\usetheme{Goettingen}
	%\usetheme{Hannover}
	%\usetheme{Ilmenau}
	%\usetheme{JuanLesPins}
	%\usetheme{Luebeck}
	\usetheme{Madrid}
	%\usetheme{Malmoe}
	%\usetheme{Marburg}
	%\usetheme{Montpellier}
	%\usetheme{PaloAlto}
	%\usetheme{Pittsburgh}
	%\usetheme{Rochester}
	%\usetheme{Singapore}
	%\usetheme{Szeged}
	%\usetheme{Warsaw}
	
	% As well as themes, the Beamer class has a number of color themes
	% for any slide theme. Uncomment each of these in turn to see how it
	% changes the colors of your current slide theme.
	
	%\usecolortheme{albatross}
	%\usecolortheme{beaver}
	%\usecolortheme{beetle}
	%\usecolortheme{crane}
	%\usecolortheme{dolphin}
	%\usecolortheme{dove}
	%\usecolortheme{fly}
	%\usecolortheme{lily}
	%\usecolortheme{orchid}
	%\usecolortheme{rose}
	%\usecolortheme{seagull}
	%\usecolortheme{seahorse}
	%\usecolortheme{whale}
	%\usecolortheme{wolverine}
	
	%\setbeamertemplate{footline} % To remove the footer line in all slides uncomment this line
	%\setbeamertemplate{footline}[page number] % To replace the footer line in all slides with a simple slide count uncomment this line
	
	%\setbeamertemplate{navigation symbols}{} % To remove the navigation symbols from the bottom of all slides uncomment this line
}

\usepackage{booktabs} % Allows the use of \toprule, \midrule and \bottomrule in tables
%%%%% Packages %%%%%
\usepackage{algorithm}
\usepackage{csquotes}
\usepackage[round]{natbib}
\bibliographystyle{agsm}
\usepackage{algorithmic}
\usepackage{pgfpages}
\usepackage{ragged2e}
\usepackage{etoolbox}
\usepackage{lipsum}
\usepackage{subfig}
\apptocmd{\frame}{}{\justifying}{} 
\usepackage{xcolor}
\usepackage{dsfont}
\usepackage{tikz}
\usepackage{amsmath}
\usepackage{graphicx}
\usepackage{subfig}
\usepackage{mathtools}
\usepackage{xcolor}
\usepackage{comment}
\usepackage{appendixnumberbeamer}
%%%%% New Commands %%%%
\newcommand{\btheta}{\boldsymbol{\theta}}
\newcommand{\x}{\mathbf{x}}
\newcommand{\SUN}{\textrm{SUN}}
\newcommand{\X}{\mathbf{X}}
\newcommand{\T}{\textrm{T}}
\newcommand{\A}{\mathcal{A}}
\newcommand{\I}{\mathds{1}}
\newcommand{\hist}{\mathbb{H}_{t-1}}
\newcommand\myeq{\stackrel{\mathclap{\normalfont\mbox{def}}}{=}}
\def\app#1#2{%
	\mathrel{%
		\setbox0=\hbox{$#1\sim$}%
		\setbox2=\hbox{%
			\rlap{\hbox{$#1\propto$}}%
			\lower1.1\ht0\box0%
		}%
		\raise0.25\ht2\box2%
	}%
}
\def\approxprop{\mathpalette\app\relax}
%\pgfpagesuselayout{1 on 1}[a4paper,border shrink=5mm]

%----------------------------------------------------------------------------------------
%	TITLE PAGE
%----------------------------------------------------------------------------------------

\title[AirBnB in New-York - Exploratory Data Analysis]{Modeling Price and Popularity of AirBnB listings in New-York} % The short title appears at the bottom of every slide, the full title is only on the title page

\author[Jiang, Morsomme, Nwankwo]{Melody Jiang \and Raphael Morsomme \and Ezinne Nwankwo}
\institute[Stat 723]{Case Study 2 - Stat 723}
\date{\today} % Date, can be changed to a custom date

\begin{document}
	
	\begin{frame}
	\titlepage % Print the title page as the first slide
	\end{frame}

\begin{frame}
	\frametitle{Overview} % Table of contents slide, comment this block out to remove it
	\tableofcontents % Throughout your presentation, if you choose to use \section{} and \subsection{} commands, these will automatically be printed on this slide as an overview of your presentation
\end{frame}

%----------------------------------------------------------------------------------------
%	PRESENTATION SLIDES
%----------------------------------------------------------------------------------------


%%%%%%
\section{Introduction}
\begin{frame}{Philosophy}
Dirty data limit us to simple models.

Very dirty data:

- "last day of the month" type of data: price, frequency of booking, constant over time?

- improbable values: minimum length $> 1,000$.

- even the dependent variables are shaky: $ popularity = number booking / availabiliy$ ?.

- ideally, \textit{tidy} data (Wickham, 2009) with one row per booking

Focus on data cleaning and feature engineering over modeling.

EDA will motivate the creation of new variables and the cleaning of the data.
\end{frame}



\section{EDA}

\begin{frame}{EDA - A City of Two Tales}
<Figure (histogram/density) showing short vs. long stay>
\end{frame}

\begin{frame}{EDA - Are you available?}
<Figure (histogram/density) showing distribution of most recent stay>
\end{frame}

\begin{frame}{EDA - Attractions}
<Map showing effect of an attraction on price>
\end{frame}


%%%%%%
\section{Data Cleaning}

\begin{frame}{Data Cleaning}
Drawing on the EDA, focus on \underline{active} listings for \underline{short stay}:

Keep listings with

(i) last review < 1 year old [lose $15,000$]

(ii) minimum number days $< 30$ (short type of stay) [lose $XXX$]

\end{frame}


\section{Feature Engineering}
\begin{frame}{Feature Engineering - Proximity}
EDA shows impact of attraction on price. This suggests the creation of a variable measuring the proximity of a listing to attractions. The proximity variable is defined as the average proximity of the listing to the attractions
$$proximity(X) = \dfrac{1}{n}\sum_{i=1}^{n} \dfrac{1}{dist(X, attraction_i)}
$$
where
$$ 
dist(x, y) = \mid latitude_{x} - latitutde_{y} \mid +  \mid longitude_{x} - longitude_{y} \mid.
$$
is the Manhattan distance.

Similarly, we compute the proximity to the closest metro station.
\end{frame}

\begin{frame}{Feature Engineering - Textual Data}

Sentiment analysis of listing name

- "documents" too short for topic modeling
- Afinn dictionary (gradual rating)

$$ Sentiment(X) = \dfrac{1}{n}\sum_{i=1}^{n} dictionary(x_i)
$$
where $Afinn(x) \in \left\lbrace -5, -4, \dots, 5\right\rbrace $.

Origin of host name 

- use name frequency as a proxy

\end{frame}

%%%%%%%
\section{Models}
\begin{frame}{Models}
Linear regression model $Y = X \beta$
where $X$ consists of:

proximity metro, proximity attraction, host name frequency, listing name sentiment, [newly created variables]

X1, X2, X3 [regular variable]

Random forest (n = $1,500$, m = $2/3$)

BMA (setting)
\end{frame}

\begin{frame}{Influential Factors}
\textit{Variable Importance} metric from the random forest (n = $1,500$, m = $2/3$)

\textit{Posterior Inclusion Probability} from the BMA
\end{frame}

\begin{frame}{Sensitivity Analysis}
Vary the setting of the RF: different levels of pruning, different values for $m$.

Vary the priors in the BMA: prior1, prior2, prior3
\end{frame}


%%%%%%%
\section{Results}
\begin{frame}{Results - Influential Factors}
<Table of variable importance>

<Table of PIP>
\end{frame}


\begin{frame}{Results - Q3}
<Figure for Q3>
\end{frame}




\begin{frame}
\frametitle{References}
\footnotesize{
	\begin{thebibliography}{99} % Beamer does not support BibTeX so references must be inserted manually as below
		
		\bibitem[Whickam, 2013]{Whickam2009} Whickam, H. \\
		\newblock Tidy Data\\
		\newblock \emph{Journal}, month year
		
	\end{thebibliography}
}
\end{frame}

\end{document} 